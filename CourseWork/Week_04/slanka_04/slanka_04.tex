% Document type with options
\documentclass[fleqn,letterpaper,12pt]{report}

% Required packages and settings (DO NOT UPDATE THIS)
\usepackage{UN5390}
../LaTeXTemplates/Course/UN5390_Settings.tex

% Assignment-specific setting (UPDATE THIS WHEN NECESSARY)
\week{04}

% Document begins
\begin{document}

% Title (DO NOT UPDATE THIS)
\assignmenttitle

%
\phantomsection
\addcontentsline{toc}{section}{Problem \# 1}
\problem
{\bf Red Flags Rule }\\
The Red Flags Rule requires many businesses and organizations to implement a written Identity Theft Prevention Program designed to detect the warning signs – or red flags – of identity theft in their day-to-day operations.The bottom line is that a program can help businesses spot suspicious patterns and prevent the costly consequences of identity theft.The Federal Trade Commission (FTC) enforces the Red Flags Rule with several other agencies.This article has tips for organizations under FTC jurisdiction to determine whether they need to design an identity theft prevention program.\cite{RF}\\
\\
{\bf Federal Information Security Modernization Act (FISMA)}\\
The Federal Information Security Modernization Act (FISMA) of 2014 updates the Federal Government's cybersecurity practices. It does it by Codifying Department of Homeland Security (DHS) authority to administer the implementation of information security policies for non-national security federal Executive Branch systems, including providing technical assistance and deploying technologies to such systems. It also 
Amends and clarifies the Office of Management and Budget's (OMB) oversight authority over federal agency information security practices; and by requiring OMB to amend or revise OMB A-130 to eliminate inefficient and wasteful reporting.\cite{FIMA}\\
\\
{\bf Family Educational Rights and Privacy Act (FERPA)}\\
The Family Educational Rights and Privacy Act (FERPA) (20 U.S.C. § 1232g; 34 CFR Part 99) is a Federal law that protects the privacy of student education records. The law applies to all schools that receive funds under an applicable program of the U.S. Department of Education.FERPA gives parents certain rights with respect to their children's education records. These rights transfer to the student when he or she reaches the age of 18 or attends a school beyond the high school level. Students to whom the rights have transferred are "eligible students."\cite{FE}




%
\newpage
\phantomsection
\addcontentsline{toc}{section}{Problem \# 2}
\problem
The Psuedo-code is the code for generating Prime numbers upto the number {\bf d}. The Outer loop increments values upto d and each time the inner loop divides the value d by the numbers upto a. 

The number of floating point operations can be mathematically given as $$y = 0.5*d^2 + 0.5*d ;$$
where {\bf y} is the number of floating point operations.

This method can be further improved by reducing the number of floating point operations. If we save all the prime numbers obtained thus far in an array, and divide {\bf a} with only the numbers in the array, we can reduce the number of floating point operations.

The program for the calculation of the number of floating point operations is in the {\em problem2.m} file.
%
\newpage
\phantomsection
\addcontentsline{toc}{section}{Problem \# 3}
\problem
TYPESET YOUR SOLUTION FOR PROBLEM 3

\vfill
and so on.

\newpage
% References
% http://tex.stackexchange.com/questions/84099/bibliographies-from-multiple-bib-files
\phantomsection
\addcontentsline{toc}{section}{References}
\section*{References}
\bibliographystyle{unsrt}
\bibliography{sgowtham,slanka} % REPLACE john WITH YOUR MICHIGAN TECH ISO USERNAME

% Document ends
\end{document}
