
% Document type with options
\documentclass[letterpaper,12pt,fleqn]{article}

% Required packages and settings (DO NOT UPDATE THIS)
\usepackage{UN5390}
%
% DO NOT EDIT THIS SECTION
\course{UN5390: Scientific Computing I}
\university{Michigan Technological University}
\term{Fall 2016}
\year{2016}

% EDIT THIS SECTION
\studentname{Sandeep Lanka}
\studentemail{slanka@mtu.edu}
\advisorname{Micheal Roggemann}
\advisoremail{mroggema@mtu.edu}


% Project title -- UPDATE THIS PART
\project{
Using fourth order Runge-Kutta adaptive time step method to simulate the celestial free return trajectory followed by a spacecraft in the earth-moon gravitational field

% Document begins
\begin{document}
\thispagestyle{empty}

% Title (DO NOT UPDATE THIS)
\projecttitle

% Guidelines (COMMENT THIS SECTION IN Description_${USER}.tex)
%\section*{Guidelines}

%\begin{enumerate}
%  \item Do not edit this file directly as it might be periodically
%       overwritten with changes. Copy \gfilename{Description.tex} as 
%        \gfilename{Description\_\$\{USER\}.tex}, and edit the latter.
%  \item Refer to \textsf{Tips} section in the course material for step by
%        step instructions to compile \gfilename{Description\_\$\{USER\}.tex}, 
%        and commit \gfilename{Description\_\$\{USER\}.*} to the GitHub 
%        repository.
%  \item Keep your research advisor happy by making timely and meaningful
%        progress. He/She controls the score for this project, worth 20\% of 
%        the final grade.
%\end{enumerate}

\newpage
% Introduction -- one paragraph; no more than 3-4 sentences
\section*{Introduction}
\addcontentsline{toc}{section}{Introduction}

A gravitational slingshot, gravity assist maneuver, or swing-by is the use of the relative movement and the gravity of a planet or other astronomical object to alter the path and speed of a spacecraft, typically in order to save propellant, time, and expense. Gravity assistance can be used to accelerate a spacecraft, that is, to increase or decrease its speed and/or redirect its path. The `assist" is provided by the motion of the gravitating body as it pulls on the spacecraft. e.g. The earth's gravity can be used to slingshot a satellite in a direction to reach a destination. It was used by interplanetary probes from Mariner 10 onwards, including the two Voyager probes' notable flybys of Jupiter and Saturn and in the Mars orbiter mission(MOM), `Mangalyan", to reach planet mars by ISRO. \cite{sling} \cite{mom}
% Description -- no more than two paragraphs; 
% no more than 5-6 sentences per paragraph
\section*{Description}
\addcontentsline{toc}{section}{Description}



\newpage

\begin{equation}
  \pi_{\mathrm{newton}} \:=\: 2 \:\sum_{n=0}^{\infty}\:\frac{2^{n}\:\left(n!\right)^{2}}{\left(2n+1\right)!}
  \hspace{0.50in}
  \pi_{\mathrm{madhava}} \:=\: \sqrt{12}\:\sum_{n=0}^{\infty}\:\frac{\left(-3\right)^{-n}}{2n + 1}
  \label{EQN_01}
\end{equation}


\vfill
% References
\phantomsection
\addcontentsline{toc}{section}{References}
\section*{References}
\interlinepenalty=1000
\bibliographystyle{unsrt}
\bibliography{sgowtham,slanka} % REPLACE john WITH YOUR MICHIGAN TECH ISO USERNAME

\vfill
% Document ends
\end{document}
