% Document type with options
\documentclass[letterpaper,12pt,fleqn]{article}

% Required packages and settings (DO NOT UPDATE THIS)
\usepackage{UN5390}
../LaTeXTemplates/Course/UN5390_Settings.tex

% Project title -- UPDATE THIS PART
\project{Using fourth order Runge-Kutta method to simulate the celestial free return trajectory followed by a spacecraft in the earth-moon gravitational field
}

% Document begins
\begin{document}
\thispagestyle{empty}

% Title (DO NOT UPDATE THIS)
\projecttitle

% Guidelines (COMMENT THIS SECTION IN Description_${USER}.tex)
%\section*{Guidelines}

%\begin{enumerate}
%  \item Do not edit this file directly as it might be periodically
%        overwritten with changes. Copy \gfilename{Description.tex} as 
%        \bf{Description\_\$\{USER\}.tex}, and edit the latter.
%  \item Refer to \textsf{Tips} section in the course material for step by
%        step instructions to compile \bf{Description\_\$\{USER\}.tex}, 
%        and commit \gfilename{Description\_\$\{USER\}.*} to the GitHub 
%        repository.
%  \item Keep your research advisor happy by making timely and meaningful
%        progress. He/She controls the score for this project, worth 20\% of 
%        the final grade.
%\end{enumerate}

\newpage
% Introduction -- one paragraph; no more than 3-4 sentences
\section*{Introduction}
\addcontentsline{toc}{section}{Introduction}
This project attempts to simulate the trans-lunar injection(TLI), a propulsive maneuver used to set a spacecraft on a trajectory that will cause it to arrive at the Moon with no further course correction anywhere\cite{TLI} and also the free-return trajectory where a free-return trajectory is a trajectory of a spacecraft traveling away from a primary body (for example, the Earth) where gravity due to a secondary body (for example, the Moon) causes the spacecraft to return to the primary body without propulsion\cite{Free}.


% Description -- no more than two paragraphs; 
% no more than 5-6 sentences per paragraph
\section*{Description}
\addcontentsline{toc}{section}{Description}

A gravitational slingshot, gravity assist maneuver, or swing-by is the use of the relative movement and the gravity of a planet or other astronomical object to alter the path and speed of a spacecraft, typically in order to save propellant, time, and expense.Gravity assistance can be used to accelerate a spacecraft, that is, to increase or decrease its speed and/or redirect its path. It was used by interplanetary probes from Mariner 10 onwards,and recently in the Mars orbiter mission(MOM), ``Mangalyan", to reach planet mars by ISRO.\cite{mom}

A spacecraft performs TLI to begin a lunar transfer from a low circular parking orbit around Earth. The large TLI burn, usually performed by a chemical rocket engine, increases the spacecraft's velocity, changing its orbit from a circular low Earth orbit to a highly eccentric orbit.The TLI burn is sized and timed to precisely target the Moon as it revolves around the Earth. The burn is timed so that the spacecraft nears apogee as the Moon approaches. Finally, the spacecraft enters the Moon's sphere of influence, making a hyperbolic lunar swingby\cite{sling}. 

The translunar injection(TLI) is applied at an instant, $t = 0$, the instant at which TLI takes place and at a certain velocity $v$, where $v = v_o + \Delta v$ and at an angle $ \alpha $.
The simulation would be carried out identifying the angle $ \alpha $, such that by the use of the Runge-Kutta method, the minimum travel time is predicted. The trajectory is specified to be within 1750 $km$ of lunar center, loo around the moon and return to 100 $km$ of the Earth's surface and the spacecrafts journey is expected to be complete within 7 days. 

The simulation and visualisation will be carried on MATLAB restricting to a 2D plane. HPCC platform\cite{HPCC} will be used to process data-parallel processing for calculating the apt angle $\alpha$. TLI and Free return trajectory's overview is shown in the Figure 1\cite{Free}.
\begin{figure}[ht!]
	\centering
	\includegraphics[height=120mm,width=160mm]{traj.pdf}
	\caption{TLI and Free return trajectory\label{overflow}}
\end{figure}



\vfill
% References
\phantomsection
\addcontentsline{toc}{section}{References}
\section*{References}
\interlinepenalty=1000
\bibliographystyle{unsrt}
\bibliography{sgowtham,slanka} % REPLACE john WITH YOUR MICHIGAN TECH ISO USERNAME

\vfill
% Document ends
\end{document}
